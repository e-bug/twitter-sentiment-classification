Sentiment analysis is an interesting problem aiming to give a machine the ability to understand the emotions and opinions expressed by humans. This is an extremely challenging task due to the complexity of human language, which makes use of rhetorical devices such as sarcasm or irony.\\
Twitter is a popular ``micro-blogging" social networking website for conveying opinions and thoughts, and thus a successful sentiment classification model based on Twitter data could provide interesting trends regarding prominent topics in the news or popular culture. For example, one could gauge the popular opinion of a politician by calculating the sentiment of all tweets containing the politician's name. Sentiment analysis in Twitter is a significantly different paradigm than past attempts at sentiment analysis through machine learning since its users are only allowed to post short status updates of less than 140 characters (``tweets"). Moreover, as in most online social networks, users create their own words and spelling shortcuts, that, in addition to misspellings, slang, and abbreviations, make this task even more challenging.\\
The aim of this project is to build an accurate sentiment analyzer for tweets. 
That is, given a user-generated status update, our classification model determines whether the given tweet reflects a positive or negative opinion on the user’s behalf.\\
In this report, we will discuss about the methods for building such sentiment analyzer including data pre-processing, creation of word vector representations and hyperparameter optimization for different classifiers.